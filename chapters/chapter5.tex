\chapter{Conclusion}
\label{chapter5}
We have presented several gimbal control algorithms: angle commanding gimbal control, angular velocity commanding gimbal control, and adaptive dxepth gimbal control. Although they have their own strengths and weaknesses, the adaptive depth gimbal control is the most useful when a depth sensor is not available and the camera is on a moving platform. Thus, the algorithm is more suitable than the other algorithms on small UAV tracking application. 

Also, a novel UAV control algorithm for target tracking, the unit vector UAV visual servoing, has been developed. This algorithm employs some non-linear control techniques to eliminate the need for the UAV altitude or target depth measurements to be used in the control algorithm. Thus, theoretically this algorithm would perform better when target is moving on non-flat surface than the algorithm using simple camera geometry. However, the simulation and hardware comparison remain as future work.

In addition, this thesis shows the first attempt to use the visual multiple target tracking using recursive RANSAC to close the UAV control loop in real-time. Although the theoretical contribution is not huge, the value sits in regard of the system integration and hardware result. 

Here are some suggestions and recommendations that can be pursued in the future.
\begin{itemize}
	\item Hardware test for the unit vector UAV visual servoing
	\item Integrate the unit vector UAV visual servoing algorithm in the autonomous target following system
\end{itemize}
